%personel data
\title{Differentialgeometrie I}
\author{Dr. Anna Siffert}
\date{Sommersemester 2018}


%math and theorems
\usepackage{amsmath}
\usepackage{amssymb}
\usepackage[amsmath,thmmarks,hyperref]{ntheorem}

%language settings and microtype
\usepackage{fontspec} 
\setmainfont{Palatino}
\setsansfont{Optima}
\setmonofont[Scale=MatchLowercase]{Menlo}

\usepackage{polyglossia}
\setmainlanguage{german}
\setotherlanguages{english}
\usepackage{microtype}

%useful packages
\usepackage{geometry}
\usepackage{xcolor}
\usepackage{graphicx}
\usepackage{float}
\usepackage{fancyhdr}
\usepackage{csquotes}
\usepackage{blindtext}
\usepackage{todonotes}


%geometry
\geometry{
	width=150mm,
	bindingoffset=7mm,}

%color settings
\definecolor{myred}{RGB}{196,19,47} 
\definecolor{myblue}{RGB}{0,139,139}


%for empty pages at the beginning of the document
\def\blankpage{%
	\clearpage%
	\thispagestyle{empty}
	\addtocounter{page}{-1}
	\null%
	\clearpage}

% Page layout for beginning of chapter
\fancypagestyle{plain}{%
	\fancyhf{}  % clear all header and footer fields 
	\fancyfoot[C]{- \thepage\hspace{3pt}-} 
	\renewcommand{\headrulewidth}{0pt}
	\renewcommand{\footrulewidth}{0pt}}

%page layout for the other pages
\pagestyle{fancy}
\fancyhf{}
\fancyhead[LE]{\textit{\nouppercase{\leftmark}}}
\fancyhead[RO]{\textit{\nouppercase{\rightmark}}}
\fancyfoot[C]{- \thepage\hspace{3pt}-}
\renewcommand{\headrulewidth}{0.2pt}
\renewcommand{\footrulewidth}{0pt}

% titles and stuff
\setkomafont{chapter}{\normalfont\bfseries\huge}
\setkomafont{section}{\normalfont\bfseries\LARGE}
\setkomafont{subsection}{\normalfont\bfseries\large}
\setkomafont{title}{\normalfont\bfseries\Large}
\setkomafont{author}{\normalfont\bfseries\Large}
\setkomafont{date}{\normalfont\bfseries\Large}

%nice boxes
\usepackage[many]{tcolorbox}    
\newtcolorbox{mybox}[1]{%
	tikznode boxed title,
	enhanced,
	arc=0mm,
	interior style={white},
	attach boxed title to top center= {yshift=-\tcboxedtitleheight/2},
	fonttitle=\large\bfseries,
	colbacktitle=white,coltitle=black,
	boxed title style={size=normal,colframe=white,boxrule=0pt},
	title={#1}}

%proofs,defs ...

%Theorems
\theoremstyle{marginbreak}
\theoremheaderfont{\normalfont\bfseries}
\theorembodyfont{\slshape} 
\theoremseparator{}
\newtheorem{satz}{Satz}[chapter]

%Lemma
\theoremstyle{changebreak} 
\theoremindent0.5cm 
\theoremseparator{}
\newtheorem{lem}[satz]{Lemma}

%Corolarys
\theoremindent0cm 
\theoremseparator{}
\newtheorem{kor}[satz]{Korollar}

%Examples
\theoremstyle{break} 
\theorembodyfont{\upshape} 
\theoremseparator{} 
\newtheorem{bsp}[satz]{Beispiel}

%Comments; Mathieu, do your design -> done ;)
\theoremstyle{plain} 
\theorembodyfont{\upshape} 
\theoremseparator{:} 
\newtheorem*{bem}[satz]{Bemerkung}

%Definitions
\theoremstyle{break}
\theorembodyfont{\slshape} 
\theoremseparator{} 
\newtheorem{defs}[satz]{Definition}

%Proofs
%\theoremheaderfont{\normalfont} %falls es nicht dick gedruckt sein soll
\theoremstyle{nonumberbreak}
\theoremseparator{} 
\theoremsymbol{\ensuremath{\square}}
\newtheorem{bew}[satz]{Beweis:}

%bibliography
\usepackage[
style=numeric-comp,
backend=biber,
maxnames=10,
maxcitenames=2,
isbn=false,
date=year,
url=false,
doi=false
]{biblatex}
\bibliography{bibliography/diffgeo-bib}

%appendix
\usepackage[toc,page]{appendix}

%always use hyperref at the end of the preamble!
\usepackage[colorlinks=True]{hyperref}
\hypersetup{allcolors=myred}
